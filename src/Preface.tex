\chapter*{Preface}
%% Add Chapter to the TOC in Italics
\addcontentsline{toc}{chapter}{\textit{Preface}}

I should start by saying that CommonUnits started as a chat in a Developers community on Telegram.
Devs Chat is one of the groups in the \fhref{TheDevs}{https://t.me/thedevs} network. See their list of groups at \url{https://thedevs.network}.

\section*{Why CommonUnits?}
The International System of Units (Système international/SI) did an excellent job at defining standard units for use in scientific and daily use. They built on top of the metric system and set in stone (or metal, in the case of the kilogram) the units that were to be followed for decades to come.
The SI system was great when it was published in 1960.
However, today, as some of us look back and say, ``Why is time 60 seconds in a minute instead of 100?'', ``Why is 11:50 not midway from 11 to lunchtime?'', it might be worth thinking of an alternative system—one set entirely in base 10.

\section*{Why base 10?}
Base 10 is the most intuitive for humans to understand. The reason is straightforward: we have ten fingers to count on. As a child at school, I remember me and my batch being unable to understand why some units were different from others. As we grew older, we have come to get used to the weirdness of our units—but that does not mean we should settle for what exists. We might as well be still running MS-DOS on IBM clones with that mindset.

\section*{The quirks of SI}
SI has seven base units—metre, kilogram, second, ampere, kelvin, mole, and candela.
\\~\\
One metre is divided into 1000 millimeters.\\
One kilogram is divided into 1000 grams (not milligrams?).\\
60 seconds make a minute (wait, why?).\\
One ampere is divided into 1000 milliamperes.\\
One kelvin is merely a kelvin.\\
One mole is the amount of substance that contains as many (arbitrary) representative particles as in 12 grams of Carbon-12.\\
One candela is divided into 1000 millicandelas.
\\~\\
Likewise, we have the following derived units:
\begin{itemize}
    \item radian (angle)
    \item hertz (frequency)
    \item newton (force, weight)
    \item pascal (pressure, stress)
    \item joule (energy, work, heat)
    \item watt (power, radiant flux)
    \item coulomb (electric charge or quantity of electricity)
    \item volt (voltage  (electrical potential difference), electromotive force)
    \item farad (capacitance)
    \item ohm (electrical resistance, impedance, reactance)
    \item siemens (electrical conductance)
    \item weber (magnetic flux)
    \item tesla (magnetic flux density)
    \item henry (inductance)
    \item degree Celsius    (temperature)
    \item lumen (luminous flux)
    \item lux (illuminance)
    \item becquerel (radioactivity  (decays per unit time))
    \item gray (absorbed dose  (of ionizing radiation))
    \item sievert (equivalent dose  (of ionizing radiation))
    \item katal (catalytic activity).
\end{itemize}

A decimal time system is not unheard of. The French and Chinese have \fhref{done it}{https://en.wikipedia.org/wiki/Decimal_time}. It just never became a global standard. Similarly, \fhref{gradian}{https://en.wikipedia.org/wiki/Gradian} is an alternative to degrees or radians in angle measurement. 100 gradians make a quadrant, and 400 make a circle.

Maybe those mentioned above were not perfect on their own either. However, we have to consider how best to solve the problems of moving from Frankenstein-like units to pure base 10. Several other units mentioned above have their quirks which at the time of creation of the SI, might have been as far as they could get. However, the SI also defines ten \fhref{metric prefixes}{https://en.wikipedia.org/wiki/Metric_prefix}. If you are going to define standard metric prefixes, might as well make full use of them.

\section*{Goals}
Our goal is to try to renormalise as many of these units to base 10 and use more intuitive standards. The \fhref{CommonUnits}{https://github.com/commonunits/draft} manifesto needs help making, and every bit is appreciated. Raise \fhref{issues}{https://github.com/commonunits/draft/issues/new/choose} and start \fhref{pull requests}{https://github.com/commonunits/draft/compare} as you deem fit.

\medskip
\begin{flushright}
  Muthu Kumar
\end{flushright}
