\chapter{Challenges}
\section{Resistance}

The proposed CU System has massive challenges. Obviously.\\
The first of them is general criticism and resistance. \textbf{Obviously}.\\
As a race, we've always \textbf{tried} to resist change, for good or bad—but change never stops.

\section{Inertia}
The second challenge is \textit{inertia to change}. When something new gets introduced, there is a time-gap for that change to be in effect where it does the opposite of productivity.
It adds an extra overhead on things that have become easy over time. Said overhead can be countered with the argument that \textbf{after} a new system is adopted, it would make things easier than the old system ever did.

Defining a bunch of units is the easier part. Fathoming its implications is an entirely different matter.
Every single textbook would need to change to adapt to a new system if one was to be created.
Practising mathematicians and physicists would curse the creators of such new standards.
There needs to be a graceful fallback or downgrade to SI\@.
All this sounds very simple, but as I write, my heart trembles.

Take the speed of light in space for example. It will not be \(3\cdot10^8m/s\) anymore (or \(299,792,458m/s\) if you want to be accurate).\\
Light travels 25,902,068,371,200 metres in 86,400 seconds or a regular day.\\
Make the following assumptions:

A regular day is 10 kiloclarkes.

A kiloclarke is 100 hectoclarkes.

A hectoclarke is 100 clarkes (After \fhref{Arthur C. Clarke}{https://en.wikipedia.org/wiki/Arthur\_C.\_Clarke}, influential science fiction author, affectionately called Prophet of the Space Age).\\
Therefore, a day is simply 100,000 clarkes.\\
The speed of light in CU is now \(2.59\cdot10^8 metres/clarke\) metres per clarke instead.
Such a change would probably face criticism because by convention, we have learnt the speed of light as \(3\cdot10^8m/s\).
CommonUnits is however convenient because it equals to \(2.59\cdot10^{13}\) metres in a day. Observe how only the order of magnitude changes while the mantissa/significand stays the same (approximated to two decimals in mantissa):
\\
The speed of light in SI\@:

\(2.99\cdot10^8 m/second\)

\(1.79\cdot10^{10}m/minute\)

\(1.07\cdot10^{12}m/hour\)

\(2.59\cdot10^{13}m/day\)
\\~\\
The speed of light in CU\@:

\(2.59\cdot10^8 m/clarke\)

\(2.59\cdot10^{10} m/hectoclarke\)

\(2.59\cdot10^{12} m/kiloclarke\)

\(2.59\cdot10^{13} m/day\)
\\~\\
\textit{Disclaimer: The above is cited purely as an example, and does not attempt to define the standard for clarkes.}

\section{Practicality}
This is not to say that it's all roses and unicorns in the land of decimals.
Some units are more resilient to change than others.
Angles are a particular case. Measuring angles in decimals is by far one of the hardest things to tackle.
\textit{Gradians} is a very reasonable way to do it—100 gradians in a right angle and 400 in a circle. The disadvantage is that 2 of 5 standard angles would become repeating decimals.\\
0 degrees would become 0 gradians.\\
30 degrees would become \(\frac{100}{3}(33.\overline{3})\) gradians.\\
45 degrees would become 50 gradians.\\
60 degrees would become \(\frac{200}{3}(66.\overline{6})\) gradians.\\
90 degrees would become 100 gradians.\\
The pattern continues in that fashion.
Here we cannot like before, arbitrarily choose to dismiss the inconvenient angles.
For example, 30 and 60-degree angles have widespread existing application in mathematics, physics, chemistry, architecture, and countless other fields.
Even if we did hypothetically change these to \(0,25,50,75\) and \(100\), their sin and cosine values would be ridiculous.
At the moment of writing, suitable solution has yet to be realised.
